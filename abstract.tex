\begin{abstract}

	Our thesis focuses on the theoretical design and potential implementation of a data capturing system to be utilized on journal documents. The system shall identify pertinent information from the manuscript in question through parsing, and collate the identified data to specified fields pertaining to the required data for ranking by the pre-existing system of the Philippine Journal Citation Index Database, or PJCID. This is to be done in such a way as to avoid erroneous overlap, repetition or omission of data which shall be caused by widely varying formats of journals and citations.

	The ranking system  utilized by the PJCID has been fully documented and proved. The problem of the system, however, lies in the data acquisition and entry done for every new journal published. The data entry for the journal ranking system requires the users to manually encode every citation for every journal entry. The sheer amount of citations makes it tedious, time-consuming, and has the potentiality for errors and overlap.

	Our thesis aims to develop a system that is able to extract journal author information from documents of journal bibliographies and citations by parsing, then collate and relate the gathered information in such a way as to remove erroneous entries of journal information due to usage of varied citation formats. The system input shall rely upon documents from the users, which will be then parsed to extract relevant information.  The system output shall be a fully organized data of citations per entry, with accompanying data on authors, title and associated information. Any errors and its potential rate of generation found in the output of the system shall be studied in order to find the feasibility of the system.
\end{abstract}
