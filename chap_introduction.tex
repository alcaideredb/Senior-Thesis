%%%%%%%%%%%%%%%%%%%%%%%%%%%%%%%%%%%%%%%%%%%%%%%%%%%%%%%%%%%%%%%%%%%%%%%%%%%
\chapter{Introduction}
%%%%%%%%%%%%%%%%%%%%%%%%%%%%%%%%%%%%%%%%%%%%%%%%%%%%%%%%%%%%%%%%%%%%%%%%%%%
Journals publications, their entries and the authors of these entries, serve as one of important foundations of modern academic studies, serving as a convenient source of new discoveries and citation sources. However, finding the best of or even the reputable authors among the mass of entries and journals can be difficult to any reader. 

The primary method of ranking journal articles is through the amount of citations it has received from other articles. The citations serve as way to verify the validity of one paper, as not many would cite a faulty article, and to show how much acclaim and trust the proponent of the article has among the academic community. Citations also serve as auxiliary support for various informational tasks, like indexing and ranking \cite{rank} , and quality assessment \cite{quality}.

This is the reason for the several journal citation index databases that has been implemented in many academic circles, a close example of which is the Philippine Journal Citation Index Database. Authors of studies, texts and documents published in the many academic journals around the world are ranked in these databases, giving users quick access to authors whose works have been referenced and trusted by many other authors to their own published papers.

\section{Project Context}
The Philippine Journal Citation Index Database, or PJCID, is a web-based citation index database funded by the Commission on Higher Education (CHED), Republic of the Philippines, in cooperation with Ateneo de Naga University, to track publications of Philippine Journals accredited by the Journal Accreditation Service (JAS) of CHED. The database currently has records of up to 44 Philippine-based journals, more than 1000 articles in total \cite{pjcid}.

The PJCID system records data on journal citations and all its associated information, like article authors and co-authors, article title, journals, publishers, along with year of publication and even kind of citations. The data collected will then be processed and resulting conclusions shall be displayed on the PJCID website as reports pertaining to the author, publisher or journal in question. The PJCID only focuses on direct citations -   articles citing earlier documents - and not bibliographic coupling - two or more articles sharing one or more references. \cite{citation}
Data gathering for the PJCID system is relies on the manual input of data from the PJCID administrative team. Identification of necessary information from the myriad citation formats and forms is done by the people involved, and they will be the ones to input the identified data into the database. Properly identifying the information, however, has proven to be rife with difficulties, like unclear nomenclature, synonyms, and publication volume, which has been recorded with a yearly increase of 3.7%. \cite{probs}

In practice, searching articles for necessary information starts with title and author acquisition and continues with extraction of authors, titles, publishers and journals of entries of the reference section of the article. In reality, however, even advanced solutions for identifying related literature, like co-word analysis, collaborative filtering, Subject-Action-Object (SAO) structures or citation analysis do often not deliver satisfying results. \cite{sols} 

This forces the current implementation of the PJCID data acquisition process to a manual approach. This makes data acquisition time-consuming and tedious to the people involved, especially on journal articles with substantial citations used. This thesis aims to create a machine-assisted process for the data acquisition of PJCID, specifically the utilization of an automated parsing system for the extraction of necessary information from journal articles.

\section{Purpose and Description}
The purpose of this thesis is the creation of a machine-assisted process for the data acquisition of PJCID. It shall focus on the parsing of journal articles derived from pdf format documents of several Philippine-based journals, deriving information necessary for report and data generation within the PJCID ranking system, like authors, titles, and citations used, along with information regarding said citations. This information shall be extracted from the input text by way of parsing, primarily using general citation formats commonly used by the many academic journals and certain keywords, an example of which is journal names, common among citations. 

The thesis shall also study any errors that occur in the output of the machine-assisted process, and compare its rate to that of the existing manual data acquisition process of the PJCID. A comparative study on the potential error rate for the machine-assisted process will serve as evidence for the potential feasibility of machine-assisted data acquisition process.

Implementation of the process in question shall be performed by way of a citation parser to be designed and implemented by the proponents. With the main difficulty of data acquisition from citations coming from the varied and uncommon nomenclature, particular focus for the design will be on coverage of as much potential variations of citation formats as possible. This includes, but is not limited to, lack of authors, shortened names, interchange of article title and journal name, lack of date, or lack of either journal or publisher name.

\section{Objectives}
The objective of this thesis is the development of a machine-assisted data acquisition process to be utilized specifically by the PJCID system. This end objective is further expounded by the following:
\begin{itemize}
	\item To study the various formats used by journal publications, both on articles and on citations
	\item To study and formulate possible variations of the aforementioned formats, and the particular formats and identifying characteristics of such variations;
	\item To design a citation parser that can extract the necessary data from articles, utilizing the plans and formats studied;
	\item To identify erroneous data from the parsed input, and remove said errors if possible;
	\item To categorize the parsed input to the information category used by the PJCID;
	\item To create an output file based on the above that shall be compatible to the PJCID system, and;
	\item To study any errors that could not be removed from the citation parser, and compare it to error rates of the existing acquisition process: 
\end{itemize}
\vfill\eject
\section{Scope and Limitations}
The thesis shall only focus on the data acquisition process of the Philippine Journal Citation Index Database. Thusly, the parsing system to be designed and implemented by the proponents shall be designed with the PJCID system in mind only. The citation parser to be designed and implemented by the proponents shall only focus on the information necessary for ranking. Journals to be used as test inputs will be ones based within the Philippines. The comparison of error rates shall be between the designed citation parser and the existing manual data acquisition process.
